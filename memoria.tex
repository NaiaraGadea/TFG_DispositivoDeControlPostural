\documentclass[a4paper,12pt,twoside]{memoir}


% Castellano
\usepackage[spanish,es-tabla]{babel}
\selectlanguage{spanish}
\usepackage[utf8]{inputenc}
\usepackage[T1]{fontenc}
\usepackage{lmodern} % Scalable font
\usepackage{microtype}
\usepackage{placeins}
\usepackage{listings}

\usepackage[style=numeric,sorting=none]{biblatex} % Para la bibliografia
\addbibresource{bibliografia.bib} % Para la bibliografia
\usepackage{geometry} 

\RequirePackage{booktabs}
\RequirePackage[table]{xcolor}
\RequirePackage{xtab}
\RequirePackage{multirow}

% Links
\PassOptionsToPackage{hyphens}{url}\usepackage[colorlinks]{hyperref}
\hypersetup{
	allcolors = {black} % Quitar o volver a poner a 'red' o 'black'
}

% Ecuaciones
\usepackage{amsmath}

% Rutas de fichero / paquete
\newcommand{\ruta}[1]{{\sffamily #1}}

% Párrafos
\nonzeroparskip

% Huérfanas y viudas
\widowpenalty100000
\clubpenalty100000

\let\tmp\oddsidemargin
\let\oddsidemargin\evensidemargin
\let\evensidemargin\tmp
\reversemarginpar

% Imágenes

% Comando para insertar una imagen en un lugar concreto.
% Los parámetros son:
% 1 --> Ruta absoluta/relativa de la figura
% 2 --> Texto a pie de figura
% 3 --> Tamaño en tanto por uno relativo al ancho de página
\usepackage{graphicx}

\newcommand{\imagen}[3]{
	\begin{figure}[!h]
		\centering
		\includegraphics[width=#3\textwidth]{#1}
		\caption{#2}\label{fig:#1}
	\end{figure}
	\FloatBarrier
}







\graphicspath{ {./img/} }

% Capítulos
\chapterstyle{bianchi}
\newcommand{\capitulo}[2]{
	\setcounter{chapter}{#1}
	\setcounter{section}{0}
	\setcounter{figure}{0}
	\setcounter{table}{0}
	\chapter*{#2}
	\addcontentsline{toc}{chapter}{#2}
	\markboth{#2}{#2}
}

% Apéndices
\renewcommand{\appendixname}{Apéndice}
\renewcommand*\cftappendixname{\appendixname}

\newcommand{\apendice}[1]{
	%\renewcommand{\thechapter}{A}
	\chapter{#1}
}

\renewcommand*\cftappendixname{\appendixname\ }

% Formato de portada

\makeatletter
\usepackage{xcolor}
\newcommand{\tutor}[1]{\def\@tutor{#1}}
\newcommand{\tutorb}[1]{\def\@tutorb{#1}}

\newcommand{\course}[1]{\def\@course{#1}}
\definecolor{cpardoBox}{HTML}{E6E6FF}
\def\maketitle{
  \null
  \thispagestyle{empty}
  % Cabecera ----------------
\begin{center}
  \noindent\includegraphics[width=\textwidth]{cabeceraSalud}\vspace{1.5cm}%
\end{center}
  
  % Título proyecto y escudo salud ----------------
  \begin{center}
    \begin{minipage}[c][1.5cm][c]{.20\textwidth}
        \includegraphics[width=\textwidth]{escudoSalud.pdf}
    \end{minipage}
  \end{center}
  
  \begin{center}
    \colorbox{cpardoBox}{%
        \begin{minipage}{.8\textwidth}
          \vspace{.5cm}\Large
          \begin{center}
          \textbf{TFG del Grado en Ingeniería de la Salud}\vspace{.6cm}\\
          \textbf{\LARGE\@title{}}
          \end{center}
          \vspace{.2cm}
        \end{minipage}
    }%
  \end{center}
  
    % Datos de alumno, curso y tutores ------------------
  \begin{center}%
  {%
    \noindent\LARGE
    Presentado por \@author{}\\ 
    en la Universidad de Burgos\\
    \vspace{0.5cm}
    \noindent\Large
    \@date{}\\
    \vspace{0.5cm}
    Tutor: D. \@tutor{}\\ % comenta el que no corresponda
    %Tutores: \@tutor{} -- \@tutorb{}\\
  }%
  \end{center}%
  \null
  \cleardoublepage
  }
\makeatother

\newcommand{\nombre}{Naiara Gadea Rodríguez Gómez}
\newcommand{\nombreTutor}{Pedro Luis Sánchez Ortega} 
\newcommand{\nombreTutorb}{Tutor 2} 
\newcommand{\dni}{71755517W} 

% Datos de portada
\title{Dispositivo de Control Postural}
\author{\nombre}
\tutor{\nombreTutor}
\tutorb{\nombreTutorb}
\date{\today}


\begin{document}

\maketitle


\newpage\null\thispagestyle{empty}\newpage

%%%%%%%%%%%%%%%%%%%%%%%%%%%%%%%%%%%%%%%%%%%%%%%%%%%%%%%%%%%%%%%%%%%%%%%%%%%%%%%%%%%%%%%%
\thispagestyle{empty}


\noindent\includegraphics[width=\textwidth]{cabeceraSalud}\vspace{1cm}

\noindent D. \nombreTutor, profesor del departamento Ingeniería Electromecánica, área de Tecnología Electrónica.

\noindent Expone:

\noindent Que la alumna D. \nombre, con DNI \dni, ha realizado el Trabajo final de Grado en Ingeniería de la Salud titulado Dispositivo de Control Postural. 

\noindent Y que dicho trabajo ha sido realizado por el alumno bajo la dirección del que suscribe, en virtud de lo cual se autoriza su presentación y defensa.

\begin{center} %\large
En Burgos, {\large \today}
\end{center}

\vfill\vfill\vfill

% Author and supervisor
\begin{minipage}{0.45\textwidth}
\begin{flushleft} %\large
Vº. Bº. del Tutor:\\[2cm]
D. \nombreTutor
\end{flushleft}
\end{minipage}
\hfill
%\begin{minipage}{0.45\textwidth}
%\begin{flushleft} %\large
%Vº. Bº. del Tutor:\\[2cm]
%D. \nombreTutorb
%\end{flushleft}
%\end{minipage}
\hfill

\vfill

% para casos con solo un tutor comentar lo anterior
% y descomentar lo siguiente
%Vº. Bº. del Tutor:\\[2cm]
%D. nombre tutor


\newpage\null\thispagestyle{empty}\newpage




\frontmatter

% Abstract en castellano
\renewcommand*\abstractname{Resumen}
\begin{abstract}

Este trabajo se centra en la búsqueda de una solución a una necesidad real con el fin de la creación de un dispositivo de control postural para el apoyo de la mejora de la postura en diferentes colectivos. Durante el desarrollo del proyecto se han estudiado los aspectos teóricos de la postura y el control postural. Por otro lado, se ha indagado en las distintas soluciones disponibles en el mercado actual con el fin de predefinir las características básicas que debería tener el dispositivo ideado. 

Tras la obtención de la idea, se ha realizado una búsqueda de los posibles componentes y se han sopesado los componentes a incluir. Se han creado dos versiones del prototipo. En la primera versión se emplea un sensor de inclinación muy sencillo y en la segunda versión se utiliza un sensor MPU más complejo que ofrece más precisión y mayores posibilidades. En el proyecto también se propone un prototipo de interfaz de una aplicación de interacción usuario-dispositivo.

La última versión del prototipo consigue un correcto control postural. Sin embargo, al tratarse de un prototipo, existen varios puntos de mejora y en el proyecto se ofrecen diferentes líneas para seguir el proyecto en un futuro.

\end{abstract}

\renewcommand*\abstractname{Descriptores}
\begin{abstract}

Control postural, postura, prototipo, dispositivo, ayuda, aplicación, aplicación interactiva, autonomía, estudio del arte, mejora, necesidad, solución, accesibilidad, sencillez, bajo coste.

\end{abstract}

\clearpage

% Abstract en inglés
\renewcommand*\abstractname{Abstract}
\begin{abstract}
This work focuses on finding a solution to a real need in order to create a postural control device to support posture improvement in different groups. During the development of the project, the theoretical aspects of posture and postural control were studied. On the other hand, we have investigated the different solutions available in the current market in order to specify the basic features that the device should have. 

After obtaining the idea, a search of the possible components was made and the components to be incuded were weighed. Two versions of the prototype have been created. The first version uses a very simple tilt sensor and the second version uses a more complex MPU sensor that offers more precision and greater possibilities. The project also proposes a prototype app interface for well user-device interaction.

The latest version of the prototype achieves correct postural control. However, the fact of being a prototype, there are several points of improvement. Different lines have been included to further develop and improve the project in the future.


\end{abstract}

\renewcommand*\abstractname{Keywords}
\begin{abstract}
Postural control, posture, prototype, device, help, application, interactive application, autonomy, art study, improvement, need, solution, accessibility, simplicity, low cost.
\end{abstract}

\clearpage

% Indices
\tableofcontents

\clearpage

\listoffigures

\clearpage

\listoftables
\clearpage


\mainmatter
\capitulo{1}{Objetivos}

Objetivos principales del trabajo realizado.

Este apartado explica de forma precisa y concisa cuales son los objetivos que se persiguen con la realización del proyecto. Se puede distinguir entre:
\begin{enumerate}
    \item Los objetivos marcados por los requisitos del software/hardware/análisis a desarrollar.
    \item Los objetivos de carácter técnico, relativos a la calidad de los resultados, velocidad de ejecución, fiabilidad o similares.
    \item Los objetivos de aprendizaje, relativos a aprender técnicas o herramientas de interés. 
    \item x
    \item Este trabajo tiene como objetivo una busqueda del estado del arte de dispositivos que tienen relación con el control postural y la creación de un dispositivo lo más sencillo y completo posible que sea capaz de identificar y avisar que una persona tiene una buena o mala postura para poder así abordar y mejorar la postura con aprendizaje dinámico.
    
    \item Suponer una ayuda para la Asosciación del Parkinson de Burgos.
    
\end{enumerate}









\capitulo{2}{Introducción}

Este trabajo surge a partir de una visita a la asociación de Parkinson Burgos\cite{ParkinsonBurgos}, dónde los fisioterapeutas comunicaron que necesitaban alguna solución para mayor autonomía de estas personas en lo relativo a su control postural. Conseguir esta solución implicaría que estos profesionales no necesiten supervisar continuamente que las personas con esta afectación mantengan una postura correcta en todo momento. El hecho de mantener una postura correcta disminuirá la cantidad de lesiones o dolor que pueden derivar de una postura incorrecta.

Aunque esta necesidad se ha comunicado por parte de un colectivo concreto, es una necesidad común. No es un problema único de personas con alguna afectación, sino que todos en algún momento adoptamos una mala postura y en algunos casos de forma continuada, y, más hoy en día, en una sociedad en la que nos pasamos la mayor parte del tiempo sentados frente a una pantalla. Esta postura incorrecta puede afectarnos a largo tiempo si no adoptamos medidas para corregirla. Una mala postura puede producir un deterioro de los músculos, daños musculares, cambios en la morfología de la columna y dificultad para realizar tareas o movimientos. 

Por todo ello, se va ha enfocar el proyecto en público general.

Todo el desarrollo y los ficheros se encuentran recopilados \href{https://github.com/NaiaraGadea/TFG_DispositivoDeControlPostural}{\textbf{\textit{Aquí}}}. Para conocer su estructura de directorios se recomienda revisar el \textit{Anexo C}.

\clearpage
\section{Tabla de siglas}
% Tabla del desglose económico de costes personales.
\begin{table}[h!]
\centering
\begin{tabular}{ m{3cm} m{6cm}  } 
\hline
\cellcolor[HTML]{EFEFEF}\textbf{Siglas} & \cellcolor[HTML]{EFEFEF}\textbf{Significado}\\
\hline

\textbf{CDM}   & {Centro de Masas} \\
\textbf{CDG}   & {Centro de Gravedad} \\
\textbf{CDP}   & {Centro de Presiones} \\
\textbf{BPI}   & {Biomechanical Postural Index Scale} \\
\textbf{DOF}   & {Degrees Of Freedom} \\
\textbf{MPU}   & {Unidad de procesamiento del movimiento} \\
\textbf{IMU}   & {Unidad Inercial de Medida} \\


\hline
\end{tabular}
\caption{Resumen de siglas empleadas en el proyecto.}
\end{table}


\section{Conceptos teóricos básicos.} 

Para llevar a cabo este trabajo hay que comprender su base, en este caso el control postural y la postura, qué es, su importancia y todo lo que ello conlleva.\cite{Libro1,Libro2} 

Algunos de los conceptos básicos que se deben comprender y que están relacionados o engloban el control postural son:

\begin{itemize}
    \item \textbf{Base de sustentación} será la superficie disponible sobre la que al cuerpo apoya su peso. En la \textit{Figura  \ref{fig:persona}}\footnote{Esta imagen ha sido generada parcialmente empleando la inteligencia artificial DALL·E} se puede observar su localización en el cuerpo humano.
    
    \item \textbf{Área de apoyo} es el área sobre la que el cuerpo descarga su peso de forma efectiva. En el caso de una persona de pie, el área de apoyo será el área correspondiente con la planta de los pies.
    
    \item \textbf{Centro de masas} \cite{CDM} o CDM es el punto de actuación de fuerzas uniformes sobre el objeto. En el cuerpo humano es la posición media de todos los centros de masas de las distintas partes del cuerpo. El CDM se encuentra relacionado con el equilibrio.  En la \textit{Figura \ref{fig:persona}} se puede observar su localización.
    
    \item \textbf{Centro de presiones} o CDP es la proyección del centro de masas del cuerpo sobre la base de sustentación. En la \textit{Figura \ref{fig:persona}} se puede observar su localización.
    
    \item \textbf{Centro de gravedad} o CDG es el punto del cuerpo donde se ejerce la fuerza de gravedad que afecta a la masa del cuerpo. Existen dos métodos para encontrar el centro de gravedad, el método de balanceo\footnote{\textbf{Método de balanceo}: método que permite calcular el centro de gravedad de objetos simétricos que se basa en el equilibrio del objeto en un punto concreto} y el método de los pesos\footnote{\textbf{Método de los pesos}: método que permite calcular el centro de gravedad de objetos no simétricos, se emplean las distancias a dos puntos para calcular el centro de gravedad.}. En la \textit{Figura \ref{fig:persona}} se puede observar su localización. 

\begin{figure}[h!]
    \centering
    \includegraphics[width=0.5\textwidth]{img/IA_Persona.jpeg}
    \caption{Diagrama de la localización de la base de sustentación y los centros de masas, presiones y gravedad de una persona de pie. Fuente propia.}
    \label{fig:persona} 
\end{figure}
    
    \item \textbf{Postura} es la orientación y alineamiento del cuerpo respecto a su entorno.
    \item \textbf{Orientación postural} es la capacidad de mantener la relación entre las distintas partes del cuerpo y el entorno al realizar una actividad.
    \item \textbf{Sinergia postural} es la relación entre la contracción muscular y las rotaciones articulares para estabilizar la postura.
    \item \textbf{Balanceo postural} es el desplazamiento constante y la corrección del centro de gravedad para mantener la postura.
    \item \textbf{Límite de estabilidad} es el trayecto por el cual una persona puede realizar un movimiento sin perder su equilibrio y también puede realizar ajustes posturales. No se trata de un límite fijo, varía con el tiempo, la tarea que se esté realizando o el entorno en el que se encuentra.
\end{itemize}

La postura nace de la relación entre el entorno, el individuo y la actividad que debe realizar. Por tanto, el \textbf{control postural} se define como el control de la posición corporal en el espacio con el fin de obtener la estabilidad y la orientación que necesitamos para poder realizar las actividades diarias, profesión o aficiones. 

La \textbf{estabilidad} es la capacidad de mantener la proyección del centro de gravedad dentro de una base de sustentación, mientras que la \textbf{orientación} es la capacidad de mantener la relación adecuada entre las distintas partes del cuerpo al realizar una tarea teniendo en cuenta el entorno.

Para poder cumplir con el objetivo del control postural el cuerpo tiene que poder anticiparse, mantenerse y reaccionar. Además, el control postural requiere que interaccionen distintos sistemas del cuerpo para abarcar la estabilidad, la percepción de la orientación espacial, el alineamiento del cuerpo, la lucha contra la gravedad al realizar un movimiento y la respuesta a posibles perturbaciones, ya sean de origen sensorial o mecánico.

Principalmente, en el control postural interviene el sistema nervioso, como centro de control, manteniendo la postura y el equilibrio gracias a la recogida e interpretación de información de los receptores y a la producción de órdenes; y, el sistema musculoesquelético, ya que se requiere de una musculatura capaz de adaptarse a los cambios. Además, como seres inteligentes que somos, se utilizan experiencias previas para elaborar el esquema corporal.

Por todo ello, se va a conocer más a fondo la relación del sistema nervioso y la postura, algunas estrategias del control postural y su importancia o desarrollo.


\subsection{El sistema nervioso y la postura.} 
El sistema nervioso se compone de diferentes estructuras como son las neuronas o las células de neuroglia, que se encargan de mantener la homeostasis corporal regulando y coordinando las distintas funciones del organismo.

Asimismo, el sistema nervioso se puede dividir en sistema nervioso central, que se encuentra compuesto por la médula espinal y el encéfalo; y el sistema nervioso periférico que está compuesto por ganglios y nervios.

El sistema nervioso también se puede dividir en función del tipo de respuestas de las que se encarga: si se encarga de las respuestas involuntarias, se trata del sistema nervioso autónomo que a su vez, se divide en los sistemas simpático y parasimpático. Mientras que, si se encarga de las respuestas voluntarias del organismo, se trata del sistema nervioso somático.

Los receptores son los que se encargan de recoger la información que se envía al sistema nervioso mediante mecanismos de retroalimentación. Existen numerosos receptores y, estos elementos, se pueden encontrar en los músculos para poder detectar el movimiento. Por otra parte, se necesita también la información recogida por la vista, el sistema vestibular del oído interno o las señales procedentes de las modificaciones de presión.

Si en algún caso se producen perturbaciones, los receptores detectarán esos imprevistos y proporcionarán información acerca de las nuevas condiciones para así adaptar el tono postural. Por ejemplo, si los receptores de presión de los pies registran el desplazamiento mínimo durante la bipedestación, se transmite la información por el nervio periférico, seguido de la médula espinal, el tracto espinocerebeloso y desde el cerebelo a la formación reticular y a los núcleos vestibulares. Si por el contrario se desplaza la cabeza se crea una aceleración en dirección anterior que se registra y se transmite a los núcleos vestibulares. 

Las reacciones de equilibrio son la respuesta al control postural, los núcleos vestibulares activan la musculatura de la 'core-stability' que está formada por los músculos del suelo pélvico, los músculos profundos paravertebrales y sacrolumbares y, la musculatura abdominal y lateral. En función del tipo de desplazamiento se activará una cadena u otra, ya sea la cadena anterior, la posterior, la lateral o combinaciones de las mismas.

\subsection{Estrategias de control postural.} 
El control postural y los ajustes se dan en tronco, tobillos y caderas, de esta forma se puede mantener el equilibrio, creando la estabilidad que posibilita los movimientos al realizar distintas actividades.

Existe un modelo que define tres elementos que modifican, construyen y mantienen la postura, el \textbf{modelo de sistemas dinámicos de Bernstein}\cite{Bernstein}. Los elementos en cuestión son los factores individuales, la tarea a realizar y el entorno. 

Los factores individuales son aquellos pertenecientes a cada individuo y pueden variar su influencia con entrenamiento. Dentro de los factores individuales encontramos los elementos sensitivos que proporcionan información respecto al movimiento y la posición del centro de gravedad. Los elementos sensitivos incluyen las aferencias visuales (permiten obtener la posición respecto al entorno), el sistema somatosensitivo (incluye los receptores de los músculos, la piel u otros tejidos, dan información acerca de las variaciones de la orientación postural) y el vestibular (inclye el oído para obtener la posición de la cabeza). También encontramos los elementos motores que hacen referencia a las exigencias musculoesqueléticas (fuerza, flexibilidad o alineación de las partes del cuerpo) y neuromusculares (patrones de movimiento y contracción de los músculos)  necesarias para el ajuste postural; y, los elementos cognitivos que refieren a las necesidades psicológicas y cognitivas relacionadas con la actitud postural.

Igualmente, existen diferentes estrategias de control de la postura, como aquellas que se centran en el equilibrio, controlando las oscilaciones o balanceos espontáneos. Algunas de estas estrategias son las de controlar la correcta alineación corporal, minimizando las fuerzas gravitatorias; el suficiente tono muscular mediante el control de la resistencia de un músculo a ser estirado; el correcto tono postural, a partir del control de la fuerza de gravedad; o, el control de las reacciones posturales o de balance mediante ajustes compuestos de reacciones de equilibrio, enderezamiento y de apoyo. 

Por otro lado, la memoria implícita que se da en el cerebelo y en los núcleos basales proporciona información sobre dónde, cuánto y cómo se debe ajustar el tono postural para compensar los desplazamientos.

Otras estrategias para mantener el control postural son las que describen Shumway-Cook y Woollacott \cite{shumwayYWoollacott}, la ‘ankle strategy’ que se basa en la bipedestación mantenida por la base de sustentación pequeña que serán los tobillos, la ‘hip strategy’ que se basa en controlar los centros de masas en un desplazamiento de peso mayor y la ‘stepping strateging’ que se basa en una base de sustentación aún mayor.

\subsection{Desarrollo del control postural.} 
Las necesidades de estabilidad cambian con la tarea que se debe realizar. El desarrollo de control postural en niños se produce en tres etapas, primero, se debe desarrollar el control encefálico, después la sedestación (capacidad de sentarse) y, por último, la bipedestación.

Además, la postura y el equilibrio varían con la edad, por una parte, los niños pequeños\cite{Libro3_pediatria} no tienen suficientemente desarrolladas las aferencias sensoriales y, por otra, los adultos mayores\cite{Libro4_mayores} presentan involución cognitiva de sus estructuras cerebrales, por lo que todos ellos ven disminuido su control postural.

Asimismo, pacientes con traumatismos craneoencefálico, esclerosis múltiple, Parkinson, infartos u otras lesiones en el sistema nervioso central, pueden tener afectado alguno de elementos implicados en el control de la postura y, en consecuencia, su control postural se verá afectado. En algunos casos también influyen los factores psicológicos, una persona con depresión o un problema de atención, también puede tener dificultades para influir sobre ese control postural.


\section{Estado del arte y trabajos relacionados.}

En la actualidad existen multitud de dispositivos electrónicos \cite{dispositivos} compuestos de distintos sensores, actuadores y algoritmos cuya finalidad es algún tipo de control postural, ya sea regulándola o modificando la postura. Se podrían diferenciar según su aplicación final: 
\begin{itemize}
    \item Mejora de la estabilidad del equilibrio de las personas con alguna alteración en el control postural como pueden ser personas con parálisis cerebral, esclerosis múltiple, Parkinson… 

    \item Prevención de una mala postura en personas que se encuentran sentadas o realizando alguna actividad. 

    \item Monitorización de la postura como feedback para mejora de ergonomía en diferentes personas, por ejemplo, en atletas\cite{Deportistas_1,Deportistas_2}.
\end{itemize}

Es necesario conocer las especificaciones que requerirán los dispositivos en función de la aplicación a la que se encuentren destinados. Algunas especificaciones que se deben tener en cuenta al diseñar o elegir un dispositivo electrónico de control postural son: 
\begin{itemize}
    \item Precisión y fiabilidad de los sensores y actuadores para detectar y responder correctamente a los cambios en la postura. 

    \item Facilidad de uso y comodidad, para que se adapte a las necesidades del usuario y a su actividad. 

    \item Autonomía y conectividad del dispositivo para que se adapte a las actividades que realiza el usuario. 
\end{itemize}

Por todo ello se realiza un estudio de distintos dispositivos que existen en la actualidad cuya finalidad o funcionamiento es similar al objetivo de este proyecto. 

\subsection{Dispositivos de control postural} 
\begin{itemize}
    \item En el grupo de investigación de Redes de Neuronas Artificiales y Sistemas Adaptativos de la \textbf{Universidad de A Coruña} junto con la empresa Bioback de Puerto Rico se ha desarrollado un dispositivo inteligente no invasivo para la corrección postural utilizando estimulación vibratoria\cite{dispUAC}. Este dispositivo se basa en el uso de ‘Biofeedback’ como medio de aprendizaje, condicionando las rutinas de los individuos.  \newline El dispositivo está compuesto por una parte de hardware que incluye 3 secciones que se colocan en 3 partes de la columna vertebral (zona cervical ( vértebras C1-T1), zona torácica (vértebras T1-L1) y zona lumbar (vértebras L1-L5)) siguiendo el modelo Goodvin\cite{Goodvin}. Cada sección del hardware esta formada por un grupo de sensores IMU que están compuestos por un acelerómetro triaxial, un giroscopio triaxial y un magnetómetro triaxial, que permiten obtener una lectura completa de la inclinación, la torsión y la flexión. Los IMUs son capaces de medir de 0º a 360º en cada eje, con una resolución de 0,1. Además, el dispositivo electrónico incluye un Sistema experto que incluye un módulo de conexión Bluetooth, una tarjeta SD para guardar los datos, una batería de litio interna recargable (16 horas de autonomía) y el sistema de vibración.  \newline Los datos son procesados por un software que permite conocer en cualquier momento la posición exacta y puede reproducir el movimiento de cada sección de la columna vertebral. Por otro lado, gracias al software se puede configurar el umbral de desviación, las secciones activadas y el tiempo de activación para proporcionar la señal de retroalimentación, es decir, el estímulo vibratorio. 

    % Referencia:
    % Rabuñal, J. R., Cuevas, J., Nogueira, M., Rodríguez-Sotillo, A., Patiño, S., Rivas, A., & Pazos, A. (2015). Dispositivo Inteligente para el Aprendizaje de la Correción Postural mediante Estimulación Vibratoria. Especial Innovación, 6. Retrieved from: http://seis.es/wp-content/uploads/2018/02/Revista-109.pdf#page=6  
\begin{figure}[h!]
    \centering
    \includegraphics[width=0.4\textwidth]{img/imgUAC.png}
    \caption{Imagen del sistema de control diseñado por la Universidad de A Coruña.\cite{dispUAC}}
    \label{fig:imgUAC} 
\end{figure}


    \item \textbf{Wireless wearable T-shirt}\cite{wirelessT-shirt}: es prototipo de una camiseta inteligente permite controlar la postura ya que incluye un sensor inductivo adherido a la camiseta de licra además de un sistema de biofeedback constante gracias a la conexión con aplicación y la señal de vibración que indica una mala postura. El sensor inductivo está formado por un alambre de cobre de un milímetro de diámetro que recorre toda la camiseta en zigzag y cuyo alargamiento modifica el valor de la impedancia generando un voltaje que se puede medir, y por lo tanto, al enderezar o encoger el cuerpo se puede observar un cambio de voltaje que se puede medir y que permite diferenciar entre una buena postura o una mala postura. Dispone de un circuito complementario al que se acopla el sensor y que incluye el módulo Bluetooth, el módulo con el motor de vibración, el microprocesador y la batería. En el caso de que el algoritmo detecte una mala postura, se genera la señal vibratoria para incentivar al usuario a modificar su postura. Se ha comprobado la utilidad del dispositivo comparando las respuestas con otros métodos ópticos obteniendo resultados satisfactorios. Este proyecto es solamente un prototipo y no tiene sistema de calibrado y no tiene en cuenta que cada persona no tiene las mismas características físicas por lo que no es posible utilizar el mismo dispositivo para todas las personas. Por otro lado, al ser un prototipo no se han realizado las suficientes pruebas sobre cómo afectaría el sensor en contacto con la piel durante un tiempo de uso continuado.
    % Referencias:
    % E. Sardini, M. Serpelloni and V. Pasqui, "Wireless Wearable T-Shirt for Posture Monitoring During Rehabilitation Exercises," in IEEE Transactions on Instrumentation and Measurement, vol. 64, no. 2, pp. 439-448, Feb. 2015, doi: 10.1109/TIM.2014.2343411. Retrieved from: https://ieeexplore.ieee.org/document/6879298
\begin{figure}[h!]
    \centering
    \includegraphics[width=0.4\textwidth]{img/imgWeaTshirt.jpg}
    \caption{Imagen de la Wireless wearable T-shirt.\cite{wirelessT-shirt}}
    \label{fig:imgWeaTshirt} 
\end{figure}

    \item \textbf{Sistema Upright Go 2} \cite{UprightGo1, UprightGo2,UprightGo3,UprightGo4,UprightGo5,UprightGo6}: es un dispositivo que se adhiere entre los omóplatos de la persona y controla la postura, enviando una vibración cuando se detecta una mala postura, para que el usuario modifique su postura. Está principalmente diseñado para personas que se encuentran sentadas frente a una pantalla. Se basa en el uso de 2 sensores, de movimiento y giroscopios. Mide la postura 100 veces en un segundo. 
    % Referencias:
    % https://store.uprightpose.com/products/upright-go2  
    % https://www.amazon.es/UpRight-Dispositivo-corrector-postura-corporal/dp/B0747YHYZF?th=1  
    % https://mejorconsalud.as.com/upright-dispositivo-mejorar-postura/  
    % https://www.uprightpose.com/en-de/science/ 
    % Manuales: https://manuals.plus/m/68128076f22ff3a39835113b4c030ef7df2065f3f8a573f0c54db8de4d0e5ad8_optim.pdf  o  https://www.manualslib.com/manual/1312878/Upright-Go.html?page=21#manual  

    \clearpage
\begin{figure}[h!]
    \centering
    \includegraphics[width=0.4\textwidth]{img/imgUprigh2.jpg}
    \caption{Imagen de Upright Go 2.\cite{UprightGo2}}
    \label{fig:imgUpright2} 
\end{figure}
    
    \item \textbf{Hipee Smart Posture Correction}\cite{Hipee1,Hipee2}: es un dispositivo similar al sistema Upriht Go creado por la empresa Hipee\cite{HipeeEmp} y respaldado por Xiaomi\cite{Xiaomi}. Es un dispositivo que se ajusta a la parte superior de la espalda del usuario a modo de collar y que en base a un sistema de sensores indica la postura del usuario. En caso de una mala postura, al superar el ángulo (ángulos entre 5 y 20º) que se ha incluido en la aplicación durante un tiempo determinado, se emite una señal de vibración a modo de recordatorio para que la persona recupere la postura correcta. Se puede conectar con su aplicación de monitorización vía Bluetooth, pudiendo obtener un informe diario y conocer los registros de los últimos 30 días. Tiene un peso de unos 50 gramos, una autonomía de hasta 90 horas y un precio aproximado de 30€. Está principalmente diseñado para un uso mientras la persona se encuentra sentada o de pie sin moverse mucho. 
    % Referencias: 
    % https://xiaomipedia.com/p/hipee-posture-correction-device/    
    % https://www.lasexta.com/tecnologia-tecnoxplora/gadgets/este-collar-inteligente-que-vende-xiaomi-evita-malas-posturas-espalda_202010225f91862051dc2300012741ef.html 
\begin{figure}[h!]
    \centering
    \includegraphics[width=0.3\textwidth]{img/imghipee.jpg}
    \caption{Imagen de Hipee.\cite{Hipee3}}
    \label{fig:imgHipee} 
\end{figure}

    \item \textbf{Sistema SPINE 3D}\cite{SPINE3D}: es un sistema de evaluación de patologías vertebrales y alteraciones posturales usando la detección optoelectrónica tridimensional, con unas cámaras RGB infrarrojas (Cámaras ToF) que identifican de forma automática los puntos de referencia en la espalda del paciente y además permiten su modificación de forma manual. Se obtiene una adquisición 3D de la parte superior del cuerpo teniendo en cuenta la longitud, la inclinación y el desequilibrio, desviación y rotación de la columna en los planos coronal, sagital y transversal, lo que ofrece al profesional un estudio completo del paciente y que le permitirá mejorar su diagnóstico y tratamiento. Se trata de un dispositivo que se utiliza de forma estática, donde el paciente se encuentra de pie y con la espalda descubierta delante del aparato. 
    % Referencias:
    % https://www.sensormedica.com/es/spine3d/   

    \item \textbf{Sistema dinámico de medición de la presión del pie}\cite{SisDinPresionPie}: se trata de un manómetro médico, que obtiene la información basándose en el análisis de la marcha y el análisis de la presión que se ejerce sobre los pies y su distribución por la planta del pie. Para obtener la información sigue 3 pasos, primero, divide el monitor de presión del pie en 4 partes iguales donde la presión debe estar distribuida aproximadamente a un 25\% por cada parte, seguidamente, realiza 18 fotogramas por segundo del estado de la presión del pie y se comprueba que la distribución de la presión de cada pie se encuentre aproximadamente en un 30\% para la zona del talón y un 20\% para la zona delantera, por último, compara los resultados promedios de los fotogramas. Se puede usar de forma estática o de forma de dinámica para medir la postura. 
    % Referencias:
    % https://alfoots.com:5000/sub/02_sub/02_sub02.php   
\begin{figure}[h!]
    \centering
    \includegraphics[width=0.4\textwidth]{img/imgSisDin.jpg}
    \caption{Imagen del Sistema dinámico de medición de la presión del pie. \cite{SisDinPresionPie}}
    \label{fig:imgSisDin} 
\end{figure}
    \item \textbf{Sistema 3D BAK}\cite{3DBAK}: es un sistema de adquisición de la morfología del paciente que utiliza cámaras de alta resolución autocalibrables en diferentes planos para realizar reconstrucciones en 3D. Además, se miden las simetrías estructurales, la rotación de las articulaciones y la desviación de la columna vertebral y se evalúan en base a la escala BPI (Biomechanical Postural Index Scale). Se devuelve un informe imprimible con la información de la rotación individual de cada vértebra y las extremidades; y, el alineamiento y desviaciones de la columna en 24 indicadores BPI y sus valores, para simplificar la tarea de diagnóstico del profesional. Es un dispositivo no invasivo y el 95,8\% de los resultados son exactamente iguales a los que se obtendrían al usar rayos X. 
    % Referencias:
    % http://diasu.com/sistemi-biometrici/  

    \item \textbf{Sistema PGO}\cite{SisPGO}: es un sistema de medición de la marcha que analiza varios parámetros que afectan a la marcha, mediante el uso de cámaras colocadas en distintos ángulos, un potente software y una cinta de correr. Analiza la inclinación de la cabeza, el nivel de la postura de cabeza hacia delante, el nivel de cifosis, la diferencia de amplitud del movimiento de entre los brazos, la diferencia de altura de los lados de los hombros y de la pelvis, la diferencia entre la longitud de los pasos, la pronación y supinación y la marcha del dedo del pie. 
    % Referencias:
    % https://alfoots.com:5000/sub/02_sub/02_sub02.php   
\begin{figure}[h!]
    \centering
    \includegraphics[width=0.5\textwidth]{img/imgPGO.jpg}
    \caption{Imagen del Sistema PGO. \cite{SisPGO}}
    \label{fig:imgPGO} 
\end{figure}
    \item \textbf{Sistema GPA}\cite{SisGPA}: Es un sistema de medición corporal a partir de la presión que se ejerce en la planta del pie y el análisis de la marcha. El dispositivo utiliza distintas cámaras para analizar las vistas anterior, superior y lateral. Además, también incluye una placa con una serie de sensores de presión. Este dispositivo permite realizar una evaluación estática, una evaluación dinámica y una podoscopia.  
    % Referencias:
    % https://alfoots.com:5000/sub/02_sub/02_sub02.php 
\end{itemize}
\begin{figure}[h!]
    \centering
    \includegraphics[width=0.5\textwidth]{img/imgGPA.jpg}
    \caption{Imagen del Sistema GPA. \cite{SisGPA}}
    \label{fig:imgGPA} 
\end{figure}
\subsection{Sistemas no específicos de control postural que podrían ser empleados para el control postural} 
\begin{itemize}
    \item \textbf{Sistema PLIANCE}\cite{Pliance1,Pliance2}: es un sistema inalámbrico de medición de la presión. Utiliza una matriz de transductores capacitivos Novel para medir la distribución de la carga sobre cualquier superficie. Los elementos y las características del transductor son fabricados por la propia empresa Novel\cite{Novel} de forma personalizada en función de las necesidades de medición. Este dispositivo se puede comunicar con el PC utilizando USB, Bluetooth y también admite almacenamiento de los datos en una tarjeta SD. Se ha añadido este sistema porque en base a la distribución de la presión en los pies se también se puede determinar la postura, y un dispositivo similar podría emplearse para la solución del objetivo del proyecto. 
    % Referencias:
    % https://www.novelusa.com/products 
\begin{figure}[h!]
    \centering
    \includegraphics[width=0.5\textwidth]{img/imgPliance.png}
    \caption{Imagen del Sistema PLIANCE. \cite{Pliance1}}
    \label{fig:imgPliance} 
\end{figure}

    \item \textbf{Siren Diabetic Socks}\cite{SirenSocks1,SirenSocks2}: calcetines inteligentes diseñados para la prevención de úlceras en los pies de personas diabéticas. Este sistema incluye unos sensores de temperatura que recogen las medidas de temperaturas cada 10 segundos con una capacidad para medir entre 20ºC y 40ºC. Los datos recogidos por los sensores se envían a un equipo especializado de profesionales sanitarios que procesan esos datos. Estos calcetines están incluidos en distintos seguros de salud como es Medicare\cite{Medicare}. Estos calcetines previenen de posibles úlceras, gangrena y una posterior posible amputación. Estos calcetines monitorizan la temperatura de la parte inferior de los pies. Estos calcetines deben ser prescritos por un profesional. Consta de una aplicación móvil que emite alertas ante aumentos significativos de temperatura. De este dispositivo se podría utilizar el formato de calcetines, utilizando sensores de presión o similar en vez de sensores de temperatura, o de forma complementaria.

    % Referencias:
    % https://www.siren.care/for-patients
\begin{figure}[h!]
    \centering
    \includegraphics[width=0.5\textwidth]{img/imgSirenSocks.png}
    \caption{Imagen de los calcetines Siren Diabetic Socks\cite{SirenSocks1}}
    \label{fig:imgSirenSocks} 
\end{figure}

    \item \textbf{Dispositivos inteligentes en la práctica de yoga}\cite{YOGA}: existen distintos dispositivos inteligentes enfocados a la mejora de la práctica de yoga, monitorizando las posturas que adopta la persona durante la realización de este deporte. Por ejemplo, existen unos leggins inteligentes que incluyen sensores y un dispositivo de feedback vibratorio, Nadi X, Smart Yoga Pants, creado por la empresa Wearable X\cite{NadiX}. Otro de los dispositivos es una esterilla de yoga con sensores de presión para el control de las posturas de este deporte. 

    
\end{itemize}

En base a los dispositivos vistos se pretenderá realizar un prototipo de uso general basándose en sensores capaces de registrar la inclinación de una persona, es decir, sensores capaces de detectar una buena o mala postura. El dispositivo será inalámbrico y vendrá acompañado por una aplicación móvil que proporcionará una mejor experiencia de usuario y un mayor conocimiento de la postura del usuario. Además, el dispositivo al ser inalámbrico deberá contar con una batería de alimentacion y deberá tener la menor cantidad de cables posible para la mayor comodidad del usuario.

\capitulo{3}{Metodología}

\section{Descripción de los datos.}
Breve descripción de los datos.
En caso de tratarse de un trabajo donde los datos son muy importantes, puede haber explicaciones extra en el anexo correspondiente.
 
\section{Técnicas y herramientas.}

Esta parte de la memoria tiene como objetivo presentar las técnicas metodológicas y las herramientas de desarrollo que se han utilizado para llevar a cabo el proyecto. Si se han estudiado diferentes alternativas de metodologías, herramientas, bibliotecas se puede hacer un resumen de los aspectos más destacados de cada alternativa, incluyendo comparativas entre las distintas opciones y una justificación de las elecciones realizadas. 
No se pretende que este apartado se convierta en un capítulo de un libro dedicado a cada una de las alternativas, sino comentar los aspectos más destacados de cada opción, con un repaso somero a los fundamentos esenciales y referencias bibliográficas para que el lector pueda ampliar su conocimiento sobre el tema.

Añadir los posibles sensores, el cable USB, microcontrolador (En mi caso arduino (hardware y software en C++ simplificado)). Las aplicaciones que se han usado para la realización del proyecto... Se deberá explicar porque se han usado unos sensores u otros (si se hace una tabla comparativa mejor). 


\subsection{Aplicaciones empleadas.}

\begin{itemize}
\item \textbf{Overleaf}: es un editor LaTeX colaborativo en línea, que se emplea para la creación, edición y publicación de documentos científicos. LaTeX es una herramienta que a partir del procesamiento de un documento de texto plano compuesto por texto y comandos LaTeX por el software de TeX engine convierte los comandos LaTeX y el texto del documento en un archivo PDF profesional. Esta es la herramienta que se ha empleado para la realización de este documento. Se puede acceder a través de https://www.overleaf.com  %(Referencia: https://www.overleaf.com/learn/latex/Learn_LaTeX_in_30_minutes#What_is_LaTeX )

\item \textbf{Diagrams.net}: es una aplicación de código abierto para la realización de diagramas online, con gran cantidad de librerías de formas para su realización. Esta aplicación es la que se ha empleado para la realización de la gran mayoría de diagramas del proyecto y los prototipos de interfaz. Se puede acceder a través de: https://www.diagrams.net % Referencia: https://www.diagrams.net/doc/getting-started-editor  

\item \textbf{Tinkercad}:es una aplicación web gratuita que permite desarrollar habilidades de diseño 3D, electrónica y programación, sin necesidad de utilizar ningún tipo de software adicional. En este proyecto se ha empleado esta herramienta para la simulación del diseño del circuito electrónico, ya que permite realizar circuitos y componentes desde 0 o empleando sus circuitos predefinidos. Además, esta aplicación consta de un editor de código que permite programar las simulaciones. Se puede acceder a través de: https://www.tinkercad.com %Referencia: https://www.tinkercad.com/circuits 

\end{itemize}

\subsection{Herramientas.}
AÑADIR IMÁGENES
\begin{itemize}
\item \textbf{Arduino}:  plataforma de código abierto de electrónica basada en hardware y software simple y accesible, originalmente creada para la creación rápida de prototipos. Existen diferentes placas de Arduino con distintas funciones. Se puede introducir en el microcontrolador un conjunto de instrucciones que se quiere que realice el dispositivo mediante el uso del software de Arduino (IDE), estas instrucciones están escritas en el lenguaje de programación Arduino, que es similar a C++ pero simplificado, además se puede expandir el lenguaje utilizando distintas bibliotecas C++. Esta plataforma es muy utilizada y ha permitido crear gran variedad de proyectos. Su sencillez, accesibilidad y coste son las principales razones por las que se ha pensado en realizar el prototipo de este proyecto basado en esta plataforma. Se utilizará como microcontrolador la placa de Arduino UNO.
% Referencia: https://www.arduino.cc/en/Guide/Introduction

\item \textbf{ProtoBoard}: placa sobre la que se construyen los circuitos electrónicos, se trata de una matriz de clavijas donde se insertan los componentes electrónicos.

\item \textbf{Cableado}: permiten la conexión del circuito.

\item \textbf{Resistencias}: componentes electrónicos que limitan el flujo de energía eléctrica del circuito.

\item \textbf{Leds}: diodo que se ilumina cuando pasa una corriente eléctrica por él.

\item \textbf{Pulsador}: para poder encender o apagar el dispositivo.

\item \textbf{Cable USB}: permite introducir las instrucciones programadas en un ordenador a la placa de Arduino.

\end{itemize}


\subsubsection{Posibles sensores.}

AÑADIR IMÁGENES Y CREAR TABLA DE COMPARACIÓN
\begin{itemize}
    \item \textbf{Módulo SCA60C}: es un módulo que consta de un sensor de ángulo SCA60C y un acelerómetro N100060. Gracias a este sensor se pueden medir ángulos de 0 a 180º, con resolución de un grado, con un voltaje de entrada de 5 voltios y una tensión de salida en función del ángulo de 0,45 - 4,5 voltios. La corriente que necesita módulo ronda los 2 mA. Este módulo admite distintos rangos de medición y se utiliza para multitud de aplicaciones en las que se necesite conocer constantemente el ángulo de giro. 
    
    \item \textbf{Galgas extensiométricas y módulo HX711}: se trata de un conjunto cuyo objetivo es medir el peso, basado en un transductor de galgas extensiométricas y un módulo HX711 que actúa como amplificador de la señal y transfiere los datos al microcontrolador. La galga extensiométrica o celda de carga es un transductor que convierte la tensión generada por los cambios en la longitud de un objeto a una señal eléctrica, en función del peso que se quiere medir existen distintas celdas de carga. Mientras que el módulo HX711 consta de un amplificador y un convertidor analógico-digital HX711, que permite la amplificación de la señal producida por la galga extensiométrica. Este módulo utiliza un puente de Weahstone para convertir la fuerza aplicada en una señal analógica. Necesita una tensión de entrada de 5 voltios y el resultado se puede obtener en g, kg o Newtons. La utilización de este módulo requiere de la librería de Arduino hx711. El precio ronda los 10-15€.
    
    \item \textbf{Acelerómetro ADXL345}: es un acelerómetro micromecanizado (MEMS) capacitivo de 3 grados de Libertad (3DOF) acoplado a un bloque de memoria FIFO que almacena hasta 32 conjuntos de coordenadas. Además, es compatible con un procesador como Arduino mediante conexio por bus SPI o bus I2C. Este dispositivo permite conocer la orientación del sensor por la acción de la fuerza de gravedad basándose en la detección de la aceleración en los ejes X, Y y Z. Se trata de un dispositivo de ultra bajo consumo utilizando en funcionamiento con unos 45 $\mu$A de corriente mientras que en Stand-By solamente usa unos 0,1 $\mu$A. Necesita una tensión de alimentación de unos 2 a 3,6 voltios. El rango de medición del dispositivo es ajustable, con resolución de hasta 13 bits y sensibilidad de 40 mg/LBS.
    
    \item \textbf{IMU MPU-6050}: es un módulo de unidad de medición inercial de 6 grados de libertad (6 DOF) fabricado por Invensense, que permite conocer la posición del sensor en todo momento. Este módulo consta de un acelerómetro de 3 ejes, un giroscopio de 3 ejes, conversores analógico a digital (ADC) de 16 bits, un sensor de temperatura, un reloj de alta precisión e interrupciones programables y un procesador interno (DMP Digital Motion Porcessor). Tanto el rango del acelerómetro como del giroscopio son ajustables. Este módulo se acopla mediante un bus SPI o un bus I2C, necesita una tensión de alimentación de unos 2,4 - 3,6 voltios y consume unos 3,5 mA al tener todos sus componentes activados. Es uno de los sensores más empleados y tiene un coste de unos 6-15€.
\end{itemize}

\capitulo{4}{Conclusiones}
\textcolor{red}{
\begin{itemize}
    \item Reafirmar las afirmaciones y los objetivos del trabajo.
    \item Reiterar los puntos principales y los resultados obtenidos.
    \item Mencionar las limitaciones y las cuestiones adecuadas para el desarrollo del tema
    \item Conectar las conclusiones con la introducción y destacar la relevancia del trabajo.
    \item Proponer trabajos futuros o líneas de investigación relacionadas.
\end{itemize}
}

Actualmente vivimos en un mundo heterogéneo y lleno de tecnologías, además, desde pequeños se nos ha dicho que debemos sentarnos correctamente, del daño que puede causar llevar mochilas pesadas y cosas similares 

En este trabajo se ha conseguido un prototipo sencillo que ...

Existen todavía trabajo que hacer para obtener un dispositivo de calidad útil. En conjunto con asociaciones o profesiones que evalúen la utilidad del dispositivo.



Este trabajo sirve como ejemplo de las posibilidades y la importancia que ofrece una visita a una asociación, como es la Asociacion Parkinson Burgos, dónde se nos ha mostrado el día a día de las personas, de la mano de los profesionales, que indicaron necesidades reales y concretas. 

Sin embargo, al formar parte de la primera promoción del grado, es la primera vez que se realiza un TFG de estas características, por lo que no ha sido posible integrarse en un laboratorio en concreto, se trata de un trabajo de fin de grado muy amplio, y sí que es verdad que tenemos conocimientos generales en practicamente todas las materias, pero existen todavía varios puntos que reforzar. Con todo ello, espero que este trabajo sirva para conocer limitaciones, dificultades y posibles carencias 


\textcolor{red}{Todo proyecto debe incluir las conclusiones que se derivan de su desarrollo. Éstas pueden ser de diferente índole, dependiendo de la tipología del proyecto, pero normalmente van a estar presentes un conjunto de conclusiones relacionadas con los resultados del proyecto y un conjunto de conclusiones técnicas. }

\textcolor{red}{Creo que en conclusiones tienes que indicar que al ser la primera vez que se realiza un TFG de estas características también sirve para ver las limitaciones, dificultades y carencias con las uqe un alumno puede enfrentarse al tratar de realizar este tipo de proyectos sin integrarse en un laboratorio en concreto. }


\section{Resumen de resultados.}

\textcolor{red}{Breve resumen de los resultados. En caso de ser un trabajo muy experimental, los resultados completos pueden aparecer en su anexo correspondiente.}

Como resultado del estudio de los diferentes productos destinados al control postural en se ha intentado unir la mayor cantidad de características para obtener un dispositivo sencillo y de bajo coste.

Como consecuencia se ha realizado una primera versión del prototipo donde se empleaba un sensor más sencillo, el sensor SW520D, con el que se conseguía cumplir la función objetivo pero con sensibilidad a vibraciones por lo se decidió probar con un sensor más complejo, el sensor MPU6050, que permitía cumplir con la función de control postural además que garantizaba la obtención de datos que se pueden emplear para la realización de estadísticas.

Además, se ha diseñado un prototipo de interfaz de una aplicación qeu facilite la interacción del usuario con el dispositivo, y que permitiese un mejor monitoreo y conocimiento de la postura del usuario. En el prototipo de interfaz se plantea no solo las modificaciones de los ajustes del dispositivo, si no que también la visualización de estadísticas de la postura del usuario o la realización de juegos o ejercicios que se incluirán en la aplicación, para el desarrollo musculoésquelético de la postura reforzando aquellos músuculos relacionados con la postura y facilitando el aprendizaje muscular de una postura natural correcta, 

\section{Discusión.}
\textcolor{red}{Discusión y análisis de los resultados obtenidos.}

En la primera fase se obtiene el resultado que se esperaba, cumple con la función de control postural, pero no es muy aceptable frente a a vibraciones y no es capaz de registrar datos ya que este dispositivo únicamente trabaja como un interruptor en función de la inclinación en la que se encuentre el sensor.

Observando y analizando los fallos encontrados se realizó el segundo prototipo, la segunda versión, donde se empleaba el módulo MPU6050, este prototipo ofrece mayor precisión y datos. Aunque aún así según los componentes del sensor no está pensado para uso prologado, ya que produce deriva.

El problema de deriva se puede solucionar con un sensor algo más complejo pero similar al MPU6050, el MPU9259 que incluye un magnetómetro que soluciona este problema. Este último sensor es el que se habría pensado para una tercera versión pero no ha sido posible realizarla por falta de tiempo.


\section{Aspectos relevantes.}
\textcolor{red}{
Este apartado pretende recoger los aspectos más interesantes del \textbf{desarrollo del proyecto}, comentados por los autores del mismo.}

\textcolor{red}{Debe incluir los detalles más relevantes en cada fase del desarrollo, justificando los caminos tomados, especialmente aquellos que no sean triviales. }

\textcolor{red}{Puede ser el lugar más adecuado para documentar los aspectos más interesantes del proyecto y también los resultados negativos obtenidos por soluciones previas a la solución entregada.}

\textcolor{red}{Este apartado, debe convertirse en el resumen de la experiencia práctica del proyecto, y por sí mismo justifica que la memoria se convierta en un documento útil, fuente de referencia para los autores, los tutores y futuros alumnos.}

\textcolor{red}{En este punto se podrían incluir los problemas que han surgido durante el desarrollo del proyecto.}

\textcolor{red}{Se han realizado diferentes versiones del prototipo a modo de kit, una primera versión con un sensor tilt (el sensor SW520D). Como relevante de esta versión es que es necesario una colocación del sensor en un determinado ángulo para que funcione correctamente, y por lo tanto sería necesario de la creación de una carcasa que guardase el sensor en ese ángulo concreto. }
\textcolor{red}{La segunda versión del prototipo se ha realizado con el módulo MPU6050. Como aspecto relevante es que se ha creado una base donde colocar el sensor y que sea sencillo de calibrar.}





%\capitulo{5}{Aspectos relevantes del desarrollo del proyecto}

\capitulo{4}{Lineas de trabajo futuras}

Al haberse realizado en este proyecto únicamente prototipos, todavía existe un gran camino que recorrer para conseguir un dispositivo robusto, preciso, útil y de bajo coste. 

El primer paso sería solucionar los problemas de calibración para no tener que calibrar cada vez que se enciende el dispositivo. Seguidamente se deberá realizar los siguientes prototipos empleando, por ejemplo, el módulo MPU-9050 para poder utilizar el dispositivo durante mayor tiempo, incluir una batería y un módulo Bluetooth o similar para liberar el dispositivo de cables que puedan molestar al usuario y dificultar las pruebas de uso.

Asimismo, se deberá crear una base de datos donde se irán recogiendo los datos proporcionados por el dispositivo. Una vez se tenga esta base de datos se podrán realizar distintos análisis y crear estadísticas útiles que permitirán obtener información relevante para el usuario del dispositivo. Además estas estadísticas y datos se podrán mostrar en la aplicación.

En el futuro se deberán ir corrigiendo posibles errores que pudieran surgir y sería muy interesante trabajar en conjunto con profesionales o asociaciones, como la asociación Parkinson Burgos\textcolor{red}{citar}, para recoger un punto de vista distinto gracias a opiniones, otras ideas y necesidades, y de esta forma aplicarlas en nuestros prototipos para poder mejorar el dispositivo.

El siguiente paso será implementar la interfaz de usuario para facilitar la interacción con el dispositivo. La aplicación deberá ser capaz de identificar distintos usuarios y conectarse al dispositivo, para poder recopilar los datos proporcionados por el usuario durante la monitorización de la postura y poder crear y mostrar diferentes estadísticas. 

Además, para complementar el proyecto se podrán investigar ejercicios que ayuden a mejorar la musculatura asociadad a la postura y a partir de esos ejercicios se podrían crear minijuegos o conjuntos de ejercicios que interaccionen con el dispositivo y que se puedan incluir en la aplicación para ayudar al usuario a mejorar el usuario.

Por último, se deberán realizar pruebas con voluntarios entregando consentimientos informados y cuestionarios de usabilidad para poder mejorar posibles errores o mejorar la experiencia de usuario. Una vez se obtenga un prototipo robusto y confiable y que cumpla con todos los objetivos y regulaciones legales, se podrá ir pensando en su comercialización. 

Por otro lado, en el futuro, se podría profundizar en otro tipo de sensores para el mismo fin, cómo los sensores de presión; finalidades como el diagnóstico de alteraciones posturales o posibilidades de dispositivos inteligentes existentes como son las mallas o calcetines inteligentes. 

En definitiva, en el futuro será necesario ir desarrollando prototipos conjuntamente con profesionales o asociaciones, y, por el camino ir resolviendo errores o añadiendo o modificando componentes, si fuera necesario, ya sea para mejora del prototipo o disminución del coste del mismo. Además, se deberán cumplir los requisitos legales y crear los documentos necesarios par poder realizar pruebas del dispositivo con distintos usuario y recopilar información gracias a cuestionarios de usabilidad. Asimismo, será necesario estar en constante investigación para descubrir posibilidades, objetivos y mejoras que añadir al dispositivo y, de esta forma obtener el dispositivo más completo y de bajo coste posible.



% -----------------BIBLIOGRAFÍA-------------------
%\bibliographystyle{plain} % Sin URL con números
%\bibliography{bibliografia}

% Si se salen de los márgenes las referencias:
%\newgeometry{left= 3cm, right = 3cm}
\printbibliography[]



\end{document}
