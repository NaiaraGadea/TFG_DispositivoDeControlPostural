\capitulo{1}{Objetivos}

Objetivos principales del trabajo realizado.

Este apartado explica de forma precisa y concisa cuales son los objetivos que se persiguen con la realización del proyecto. Se puede distinguir entre:
\begin{enumerate}
    \item \textcolor{red}{Los objetivos marcados por los requisitos del software/hardware/análisis a desarrollar.}
    \item \textcolor{red}{Los objetivos de carácter técnico, relativos a la calidad de los resultados, velocidad de ejecución, fiabilidad o similares.}
    \item \textcolor{red}{Los objetivos de aprendizaje, relativos a aprender técnicas o herramientas de interés. }
    \item x
    \item \textcolor{red}{Este trabajo tiene como objetivo una busqueda del estado del arte de dispositivos que tienen relación con el control postural y la creación de un dispositivo lo más sencillo y completo posible que sea capaz de identificar y avisar que una persona tiene una buena o mala postura para poder así abordar y mejorar la postura con aprendizaje dinámico.}
    \item \textcolor{red}{Suponer una ayuda para la Asociación del Parkinson de Burgos.}
    \item \textcolor{red}{Suponer una ayuda para distintos colectivos, basicamente para todos. } 
    \item Algo como realizar un TFG que sirva para demostrar la posibilidad de dar soluciones técnicas desde la titulación a asociaciones de personas con discapacidad como la Asociación Parkinson Burgos, mediante la realización de trabajos de fin de grado con fines sociales.
 referenciar - Asociación Parkinson Burgos www.parkinsonburgos.org
    
\end{enumerate}








