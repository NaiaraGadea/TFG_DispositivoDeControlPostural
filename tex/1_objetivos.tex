\capitulo{1}{Objetivos}
Este trabajo surge de una visita a la Asociación Parkinson Burgos\cite{ParkinsonBurgos} donde nos dieron a conocer su trabajo allí y varios problemas con los que se encontraban durante la realización del mismo. Una de las dificultades comunicadas era la de tener que estar en constante supervisión de que las personas con parkinson adoptaran mientras caminaban una postura correcta, puesto que en muchos casos no se daban cuenta por sí mismos que habían adoptado posturas incorrectas durante el desarrollo de la actividad. Por ello, la principal motivación que se persigue en este proyecto es la búsqueda de una solución sencilla, accesible y  a esta necesidad.

En este apartado se explican los objetivos que se persiguen con la realización del proyecto. Se pueden distinguir los siguientes objetivos:

\begin{enumerate}
    \item Comprender los fundamentos básicos de la postura, los beneficios de una postura natural correcta y sus modificaciones a lo largo de la vida o por la realización de ciertas actividades.
    \item Comprender el proceso completo desde la comunicación de una necesidad real y el desarrollo y creación de una idea.
    \item Estudio de los dispositivos actuales destinados al control postural y sus posibilidades.
    \item Análisis de dispositivos y componentes que puedan ser empleados en el desarrollo del dispositivo de control postural ideado del proyecto.
    \item Creación de uno o varios prototipos del dispositivo de control postural que avisen en caso de que el usuario se encuentre posicionado en una mala postura que sea sencillo, completo y de bajo coste.
    \item El dispositivo tendrá la finalidad de mejorar la postura mediante aprendizaje dinámico basado en biofeedback.
    \item Creación de un prototipo de interfaz que facilite la interacción del usuario con el dispositivo.
    \item Aunar los conocimientos sanitarios y técnicos adquiridos a lo largo del grado en Ingeniería de la salud.
    \item Análisis de costes y viabilidad legal del dispositivo propuesto.
    \item Uso de herramientas como GitHub\cite{GitHub} o Trello\cite{Trello} para el seguimiento de los documentos del trabajo al igual que para la recopilación del código de los programas y documentos realizados en el proyecto.
    \item Emplear Overleaf como editor LaTex para la creación de los documentos entregables.
    \item Emplear el entorno de Arduino para la creación del prototipo.
    \item El dispositivo ideado tiene la finalidad de ayudar a distintos colectivos.
    \item Realizar un trabajo reproducible en el futuro para que otras personas puedan realizar los prototipos gracias a los datos y conocimientos desarrollados en el proyecto.
    \item Demostrar que es posible dar soluciones técnicas desde la titulación de Ingeniería de la Salud a distintos perfiles de personas, incluyendo asociaciones de personas con discapacidad como puede ser la Asociación Parkinson Burgos\cite{ParkinsonBurgos}, gracias a la realización de trabajos de fin de grado con fines sociales.
\end{enumerate}









