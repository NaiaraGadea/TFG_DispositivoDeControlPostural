\apendice{Manual del desarrollador / programador / investigador.} % usar el término que mejor se corresponda.

\section{Estructura de directorios}

Descripción de los directorios y ficheros entregados. (De github, entiendo, o también la propia aplicación si se llega a obtener)

Los ficheros de código de la memoria y de arduino se pueden encontrar en el repositorio de GitHub en la carpeta XXX.

\section{Compilación, instalación y ejecución del proyecto}

En caso de ser necesaria esta sección, porque la compilación o ejecución no sea directa.

Para utilizar los ficheros de código de arduino se deberá utilizar el Arduino IDE y tener montado el prototipo del dispositivo e introducir en el arduino el código necesario para que el dispositivo funcione.

\section{Pruebas del sistema}
Esta sección puede ser opcional.

Se ha realizado una encuesta de validación por parte del usuario.

Puede tratarse de validación de la interfaz por parte de los usuarios, mediante escuestas o similar o validación del funcionamiento mediante pruebas unitarias.



\section{Instrucciones para la modificación o mejora del proyecto.}

Instrucciones y consejos para que el trabajo pueda ser mejorado en futuras ediciones.

Se puede crear un prototipo más robusto, utilizar un módulo Bluetooth, crear la aplicación del dispositivo para la mejora de la interación del usuario con el dispositivo.

Durante el desarrollo de la primera versión del proyecto con el sensor más sencillo, el sensor SW250D, se observó que no tiene una gran precisión para esta aplicación, además es necesario crear una carcasa donde colocar el sensor de una determinada forma para que cumpla de manera correcta su función.