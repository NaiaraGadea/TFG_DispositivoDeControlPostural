\apendice{Manual del desarrollador / programador / investigador.} % usar el término que mejor se corresponda.

\section{Estructura de directorios}

\textcolor{red}{Eliminar aquellos anexos que no se hayan realizado o entregado en la memoria final}

En el repositorio de GitHub encontraremos los siguientes ficheros y directorios:
\begin{itemize}
    \item \textbf{Carpeta img}: carpeta dónde se incluyen todas las imágenes que se han empleado en el desarrollo del proyecto.
    \item \textbf{Carpeta tex}:
    \begin{itemize}
        \item \textbf{1\_objetivos.tex}: documento LaTex que contiene la información acerca de los objetivos.
        \item \textbf{2\_introduccion.tex}: documento LaTex que contiene la información acerca de la introducción del trabajao, se incluyen conceptos teóricos básicos y el estado del arte.
        \item \textbf{3\_metodologia.tex}: documento LaTex que contiene la información acerca de la metodología empleada, dónde se incluyen descripción de los datos con los que se trabajan y técnicas y herramientas. 
        \item \textbf{4\_conclusiones.tex}: documento LaTex que contiene las conclusiones del proyecto, se puede encontrar un resumen de resultados, discusión y aspectos relevantes.
        \item \textbf{5\_lineas\_futuras.tex}: documento LaTex que contiene la información acerca de las líneas de trabajo futuras.
        \item \textbf{A\_planificacion.tex}: documento LaTex que contiene información del anexo A, donde se incluye la planificación temporal, la planificación económica y la viabilidad legal en España del trabajo.
        \item \textbf{B\_manual\_usuario.tex}: documento LaTex que contiene la información del anexo B que incluye los requisitos funcionales y no funcionales, la instalación y puesta en marcha y manuales o demostraciones prácticas.
        \item \textbf{C\_manual\_programador.tex}: documento LaTex que contiene la información del anexo C que contiene la estructura de los directorios entregados, la información acerca de la ejecución del proyecto y las instrucciones de mejora del proyecto.
        \item \textbf{D\_datos.tex}: documento LaTex que contiene la información acerca de los datos utilizados en el proyecto.
        \item \textbf{E\_diseno.tex}: documento LaTex que contiene la información acerca del diseño del prototipo realizado.
        \item \textbf{F\_requisitos.tex}: documento LaTex que contiene la información acerca los casos de uso definidos.
        \item \textbf{G\_experimental.tex}: documento LaTex que contiene la información acerca del estudio experimental realizado.
        \item \textbf{readme.txt}:
    \end{itemize}
    \item \textbf{Carpeta videos}: carpeta dónde se encuentran los vídeos de demostración del proyecto.
    \item \textbf{Carpteta Arduino}: carpeta dónde se encuentran los programas de arduino empleados en las distintas versiones del prototipo del proyecto.
\begin{itemize}
    \item \textbf{ProgramaV1.ino}: Programa que se ha de cargar en la placa de Arduino de la primera versión del prototipo del dispositivo de control postural, dónde se emplea el sensor SW520D.
    \item \textbf{ProgramaV2.ino}: Programa que se ha de cargar en la placa de Arduino de la segunda versión del prototipo del dispositivo de control postural, dónde se emplea el módulo MPU-6050.
\end{itemize}
    
    \item \textbf{README.md}: archivo de presentación del repositorio de GitHub.
    \item \textbf{anexos.pdf}: documento PDF que contiene los anexos completos.
    \item \textbf{anexos.tex}: archivo LaTex que contiene la estructura del documento pdf de los anexos.
    \item \textbf{bibliografia.bib}: archivo que recoge la bibliografía de la memoria.
    \item \textbf{bibliografiaAnexos.bib}: archivo que recoge la bibliografía de los anexos.
    \item \textbf{memoria.pdf}: documento PDF que contiene la memoria completa.
    \item \textbf{memoria.tex}: archivo LaTex que contiene la estructura del documento de la memoria.
\end{itemize}


\section{Compilación, instalación y ejecución del proyecto}

\textcolor{red}{En caso de ser necesaria esta sección, porque la compilación o ejecución no sea directa.}

\textcolor{red}{Para utilizar los ficheros de código de arduino se deberá utilizar el Arduino IDE y tener montado el prototipo del dispositivo e introducir en el arduino el código necesario para que el dispositivo funcione.}

\textcolor{red}{YA SE HA EXPLICADO EN EL B2.INSTALACIÓN Y PUESTA EN MARCHA. VER SI MOVER}

\section{Pruebas del sistema}
\textcolor{red}{Esta sección puede ser opcional.}

\textcolor{red}{Se ha realizado una encuesta de validación por parte del usuario.}

\textcolor{red}{Puede tratarse de validación de la interfaz por parte de los usuarios, mediante escuestas o similar o validación del funcionamiento mediante pruebas unitarias.}



\section{Instrucciones para la modificación o mejora del proyecto.}

\textcolor{red}{Instrucciones y consejos para que el trabajo pueda ser mejorado en futuras ediciones.}

\textcolor{red}{Se puede crear un prototipo más robusto, utilizar un módulo Bluetooth, crear la aplicación del dispositivo para la mejora de la interación del usuario con el dispositivo.}

\textcolor{red}{Durante el desarrollo de la primera versión del proyecto con el sensor más sencillo, el sensor SW250D, se observó que no tiene una gran precisión para esta aplicación, además es necesario crear una carcasa donde colocar el sensor de una determinada forma para que cumpla de manera correcta su función.}


\textcolor{red}{Mejora haciendo que la versión 2 del prototipo con el módulo MPU6050 se calibre automáticamente, sin tener que clicar ningun boton.}

\textcolor{red}{Realizar un prototipo con el modulo MPU9250 o similar porque sera similar que el MPU6050 pero más preciso y sin problemas de deriva durante el uso continuado.}

Este proyecto es básicamente un prototipo, por lo que hay bastantes aspectos en los que mejorar en el futuro.

Durante el desarrollo y las pruebas de la primera fase, dónde se empleaba el sensor SW520D, se pudo observar que, aunque no se trata de un sensor de gran precisión se pudieron analizar las posibilidades de mejora y las ventajas que suponía trabajar con el sensor de mayor precisión en la fase 2.