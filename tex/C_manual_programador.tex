\apendice{Manual del desarrollador / programador / investigador.} % usar el término que mejor se corresponda.

\section{Estructura de directorios}

En el repositorio de GitHub al cual se puede acceder \href{https://github.com/NaiaraGadea/TFG_DispositivoDeControlPostural}{\textbf{\textit{Aquí}}} encontraremos los siguientes ficheros y directorios:
\begin{itemize}
    \item \textbf{Carpeta img}: carpeta dónde se incluyen todas las imágenes que se han empleado en el desarrollo del proyecto.
    \item \textbf{Carpeta tex}: carpeta que incluye todos los capítulos de la memoria.
    \begin{itemize}
        \item \textbf{1\_objetivos.tex}: documento LaTex que contiene la información acerca de los objetivos.
        \item \textbf{2\_introduccion.tex}: documento LaTex que contiene la información acerca de la introducción del trabajao, se incluyen conceptos teóricos básicos y el estado del arte.
        \item \textbf{3\_metodologia.tex}: documento LaTex que contiene la información acerca de la metodología empleada, dónde se incluyen descripción de los datos con los que se trabajan y técnicas y herramientas. 
        \item \textbf{4\_conclusiones.tex}: documento LaTex que contiene las conclusiones del proyecto, se puede encontrar un resumen de resultados, discusión y aspectos relevantes.
        \item \textbf{5\_lineas\_futuras.tex}: documento LaTex que contiene la información acerca de las líneas de trabajo futuras.
        \item \textbf{A\_planificacion.tex}: documento LaTex que contiene información del anexo A, donde se incluye la planificación temporal, la planificación económica y la viabilidad legal en España del trabajo.
        \item \textbf{B\_manual\_usuario.tex}: documento LaTex que contiene la información del anexo B que incluye los requisitos funcionales y no funcionales, la instalación y puesta en marcha y manuales o demostraciones prácticas.
        \item \textbf{C\_manual\_programador.tex}: documento LaTex que contiene la información del anexo C que contiene la estructura de los directorios entregados, la información acerca de la ejecución del proyecto y las instrucciones de mejora del proyecto.
        \item \textbf{D\_datos.tex}: documento LaTex que contiene la información acerca de los datos utilizados en el proyecto.
        \item \textbf{E\_diseno.tex}: documento LaTex que contiene la información acerca del diseño del prototipo realizado.
        \item \textbf{F\_requisitos.tex}: documento LaTex que contiene la información acerca los casos de uso definidos.
        \item \textbf{G\_experimental.tex}: documento LaTex que contiene la información acerca del estudio experimental realizado.
        \item \textbf{readme.txt}: documento de información de la carpeta.
    \end{itemize}
    \item \textbf{Carpeta videos}: carpeta dónde se encuentran los vídeos de demostración del proyecto.
        \begin{itemize}
            \item DemostracionV1:  Vídeo de la demostración de la primera versión del prototipo.
            \item DemostracionV2: Vídeo de la demostración de la segunda versión del prototipo.
        \end{itemize}
    \item \textbf{Carpeta memoria\_PDF}: carpeta dónde se encuentran los documentos PDF de la memoria.
        \begin{itemize}
            \item memoria\_NaiaraRodriguez.pdf:  documento PDF que contiene la memoria completa.
            \item anexos\_NaiaraRodriguez.pdf: documento PDF que contiene los anexos completos.
        \end{itemize}
    
    \item \textbf{Carpeta Arduino}: carpeta dónde se encuentran los programas de Arduino\cite{ArduinoIDE} empleados en las distintas versiones del prototipo del proyecto.
    \begin{itemize}
        \item \textbf{ProgramaV1.ino}: Programa que se ha de cargar en la placa de Arduino de la primera versión del prototipo del dispositivo de control postural, dónde se emplea el sensor SW520D.
        \item \textbf{ProgramaV2.ino}: Programa que se ha de cargar en la placa de Arduino de la segunda versión del prototipo del dispositivo de control postural, dónde se emplea el módulo MPU-6050.
    \end{itemize}
    
    \item \textbf{README.md}: archivo de presentación del repositorio de GitHub\cite{GitHub}.
    \item \textbf{anexos.tex}: archivo LaTex que contiene la estructura del documento pdf de los anexos.
    \item \textbf{bibliografia.bib}: archivo que recoge la bibliografía de la memoria.
    \item \textbf{bibliografiaAnexos.bib}: archivo que recoge la bibliografía de los anexos.
    \item \textbf{memoria.tex}: archivo LaTex que contiene la estructura del documento de la memoria.
\end{itemize}


\section{Instrucciones para la modificación o mejora del proyecto.}

Al ser este proyecto un prototipo hay varios aspectos de mejora para el futuro desde añadir componentes al prototipo a crear una carcasa o crear una base de datos donde se guarden los datos adquiridos.

Durante el desarrollo de la primera fase se observó que el sensor empleado no era de gran precisión y por ello se decidió crear una segunda versión con un sensor de mayor precisión, el sensor MPU-6050\cite{MPU6050_1,MPU6050_2}, que supusiera una ventaja al proyecto. 

La segunda versión tiene la necesidad de realizar una calibración cada vez que se conecta el prototipo al portátil que tiene instalado el software de Arduino IDE\cite{ArduinoIDE}, como primera propuesta de mejora se podría modificar el código del programa para que no fuese necesario calibrar el dispositivo con tanta frecuencia y para que no fuese necesario presionar un el botón de calibración siempre.

El sensor resultó exitoso, sin embargo, por las características del sensor MPU6050 este prototipo no puede ser utilizado durante largos periodos de tiempo ya que se crea deriva y se distorsionan las mediciones. Por ello, como propuesta de mejora se podrían estudiar el uso de otros sensores similares que resolviesen este problema uno de los sensores que se podrían emplear sería el MPU9250\cite{MPU9250_1,MPU9250_2}.

Además, para el prototipo se podría crear una placa electrónica donde se encuentren todos los componentes del prototipo sin la necesidad de emplear una placa de pruebas. Asimismo, se podría crear una carcasa, por ejemplo, empleando impresión 3D que protegiese el hardware y que se adaptase de manera ergonómica al cuerpo de la persona.

Al prototipo se podrían incluir otros elementos como pueden ser una batería y un módulo Bluetooth para mejorar la interacción con el usuario, eliminando cables que pueden ser incómodos para el usuario además que pueden limitar la libertad de los movimientos. Disponer de un módulo Bluetooth también permitiría transmitir los datos que recoge el sensor sin necesidad de conexión USB. 

Este último punto nos lleva a otro punto a mejorar, el almacenamiento y tratamiento de los datos. Se puede implementar una forma de almacenaje de los datos para que se pueda trabajar con ellos obteniendo distintas estadísticas que serían útiles en el futuro. Y que se realice un trabajo estadístico de esos mismos datos para conecer más información a parte de la corrección en tiempo real de la postura.
