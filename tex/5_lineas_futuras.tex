%\capitulo{5}{Aspectos relevantes del desarrollo del proyecto}

\capitulo{4}{Lineas de trabajo futuras}

Al haberse realizado en este proyecto únicamente prototipos, todavía existe un gran camino que recorrer para conseguir un dispositivo robusto, preciso, útil y de bajo coste. 

El primer paso sería solucionar los problemas de calibración para no tener que calibrar cada vez que se enciende el dispositivo. Seguidamente se deberá realizar los siguientes prototipos empleando, por ejemplo, el módulo MPU-9050 para poder utilizar el dispositivo durante mayor tiempo, incluir una batería y un módulo Bluetooth o similar para liberar el dispositivo de cables que puedan molestar al usuario y dificultar las pruebas de uso.

Asimismo, se deberá crear una base de datos donde se irán recogiendo los datos proporcionados por el dispositivo. Una vez se tenga esta base de datos se podrán realizar distintos análisis y crear estadísticas útiles que permitirán obtener información relevante para el usuario del dispositivo. Además estas estadísticas y datos se podrán mostrar en la aplicación.

En el futuro se deberán ir corrigiendo posibles errores que pudieran surgir y sería muy interesante trabajar en conjunto con profesionales o asociaciones, como la asociación Parkinson Burgos\textcolor{red}{citar}, para recoger un punto de vista distinto gracias a opiniones, otras ideas y necesidades, y de esta forma aplicarlas en nuestros prototipos para poder mejorar el dispositivo.

El siguiente paso será implementar la interfaz de usuario para facilitar la interacción con el dispositivo. La aplicación deberá ser capaz de identificar distintos usuarios y conectarse al dispositivo, para poder recopilar los datos proporcionados por el usuario durante la monitorización de la postura y poder crear y mostrar diferentes estadísticas. 

Además, para complementar el proyecto se podrán investigar ejercicios que ayuden a mejorar la musculatura asociadad a la postura y a partir de esos ejercicios se podrían crear minijuegos o conjuntos de ejercicios que interaccionen con el dispositivo y que se puedan incluir en la aplicación para ayudar al usuario a mejorar el usuario.

Por último, se deberán realizar pruebas con voluntarios entregando consentimientos informados y cuestionarios de usabilidad para poder mejorar posibles errores o mejorar la experiencia de usuario. Una vez se obtenga un prototipo robusto y confiable y que cumpla con todos los objetivos y regulaciones legales, se podrá ir pensando en su comercialización. 

Por otro lado, en el futuro, se podría profundizar en otro tipo de sensores para el mismo fin, cómo los sensores de presión; finalidades como el diagnóstico de alteraciones posturales o posibilidades de dispositivos inteligentes existentes como son las mallas o calcetines inteligentes. 

En definitiva, en el futuro será necesario ir desarrollando prototipos conjuntamente con profesionales o asociaciones, y, por el camino ir resolviendo errores o añadiendo o modificando componentes, si fuera necesario, ya sea para mejora del prototipo o disminución del coste del mismo. Además, se deberán cumplir los requisitos legales y crear los documentos necesarios par poder realizar pruebas del dispositivo con distintos usuario y recopilar información gracias a cuestionarios de usabilidad. Asimismo, será necesario estar en constante investigación para descubrir posibilidades, objetivos y mejoras que añadir al dispositivo y, de esta forma obtener el dispositivo más completo y de bajo coste posible.

