\apendice{Documentación de usuario}

\section{Requisitos software y hardware para ejecutar el proyecto.}
%Añadir requisitos y tablas de requisitos, pensar si se quieren dividir en requisitos funcionales y no funcionales.
\subsection{Requisitos funcionales}

%%%%%%%%%%%%%%%%%%%%%%%%%%%%% RF-01 %%%%%%%%%%%%%%%%%%%%%%%%%%%%
\begin{table}[h!]
\begin{tabular}{|
>{\columncolor[HTML]{EFEFEF}}c |l|}
\hline
\cellcolor[HTML]{B9E3F0}\textbf{RF-01} & \multicolumn{1}{c|}{\cellcolor[HTML]{B9E3F0}\textbf{Aplicación}}                                                                                                                                                                                \\ \hline
Descripción                            & \begin{tabular}{p{10.5cm}}La solución deberá contar con una aplicación, ya sea una aplicación de escritorio, web o móvil, para simplificar la experiencia de uso y la visualización de resultados por parte del usuario.\end{tabular} \\ \hline
Importancia                            & \cellcolor[HTML]{FFFFFF}\begin{tabular}{p{10.5cm}}Alta, es la base de la visualización del seguimiento de la persona que utiliza el dispositivo.\end{tabular}                                                                               \\ \hline
Prioridad                              & \multicolumn{1}{c|}{Alta}                                                                                                                                                                                                                       \\ \hline
\end{tabular}
\caption{Requisito Funcional 1 'Aplicación'}
\label{RF_01}
\end{table}

%%%%%%%%%%%%%%%%%%%%%%%%%%%%% RF-02 %%%%%%%%%%%%%%%%%%%%%%%%%%%%
\begin{table}[h!]
\begin{tabular}{|
>{\columncolor[HTML]{EFEFEF}}c |l|}
\hline
\cellcolor[HTML]{B9E3F0}\textbf{RF-02} & \multicolumn{1}{c|}{\cellcolor[HTML]{B9E3F0}\textbf{Iniciar grabación}}                                                                                                                                                                                \\ \hline
Descripción                            & \begin{tabular}{p{10.5cm}}La solución deberá contar con una opción de grabación, con la cual el profesional o el usuario tendrán la posibilidad de comenzar y finalizar el registro de la postura.\\ Los resultados durante la grabación se almacenarán en la plataforma para su análisis.\end{tabular} \\ \hline
Importancia                            & \cellcolor[HTML]{FFFFFF}\begin{tabular}{p{10.5cm}}Alta, ya que la grabación de las respuestas permitirá al profesional analizarlas de forma detallada con el objetivo de obtener conclusiones y determinar el grado y evolución de la afectación.\end{tabular}                                                                               \\ \hline
Prioridad                              & \multicolumn{1}{c|}{Alta}                                                                                                                                                                                                                       \\ \hline
\end{tabular}
\caption{Requisito Funcional 2 'Iniciar grabación'}
\label{RF_02}
\end{table}

%%%%%%%%%%%%%%%%%%%%%%%%%%%%% RF-03 %%%%%%%%%%%%%%%%%%%%%%%%%%%%
\begin{table}[h!]
\begin{tabular}{|
>{\columncolor[HTML]{EFEFEF}}c |l|}
\hline
\cellcolor[HTML]{B9E3F0}\textbf{RF-03} & \multicolumn{1}{c|}{\cellcolor[HTML]{B9E3F0}\textbf{Identificación de perfiles}}                                                                                                                                                                                \\ \hline
Descripción                            & \begin{tabular}{p{10.5cm}}La aplicación debe ser capaz de diferenciar a diferentes perfiles, en el caso de uso de una organización o un profesional, y una única identificación en el caso de que se trate de un usuario particular.\end{tabular} \\ \hline
Importancia                            & \cellcolor[HTML]{FFFFFF}\begin{tabular}{p{10.5cm}}Media, una vez se obtenga la base del dispositivo y su funcionamiento se puede dividir a los usuarios entre profesionales o particulares, con distintas funciones para cada uno de ellos.\end{tabular}                                                                               \\ \hline
Prioridad                              & \multicolumn{1}{c|}{Media}                                                                                                                                                                                                                       \\ \hline
\end{tabular}
\caption{Requisito Funcional 3 'Identificación de perfiles'}
\label{RF_03}
\end{table}

%%%%%%%%%%%%%%%%%%%%%%%%%%%%% RF-04 %%%%%%%%%%%%%%%%%%%%%%%%%%%%
\begin{table}[h!]
\begin{tabular}{|
>{\columncolor[HTML]{EFEFEF}}c |l|}
\hline
\cellcolor[HTML]{B9E3F0}\textbf{RF-04} & \multicolumn{1}{c|}{\cellcolor[HTML]{B9E3F0}\textbf{Detección de la postura}}                                                                                                                                                                                \\ \hline
Descripción                            & \begin{tabular}{p{10.5cm}}La solución deberá ser capaz de detectar los cambios en la postura. Para ello se deberá implementar un algoritmo que filtre en función de los datos en crudo recogidos, una postura correcta o incorrecta. Esta medición se podría obtener en forma de ‘porcentaje de buena postura’.\end{tabular} \\ \hline
Importancia                            & \cellcolor[HTML]{FFFFFF}\begin{tabular}{p{10.5cm}}Alta, dado que es la base que permitirá definir si la persona lleva una buena postura o no, y en base a ello, realizar la comunicación correspondiente y obtener las estadísticas necesarias para la toma de decisiones.\end{tabular}                                                                               \\ \hline
Prioridad                              & \multicolumn{1}{c|}{Alta}                                                                                                                                                                                                                       \\ \hline
\end{tabular}
\caption{Requisito Funcional 4 'Detección postural'}
\label{RF_04}
\end{table}

%%%%%%%%%%%%%%%%%%%%%%%%%%%%% RF-05 %%%%%%%%%%%%%%%%%%%%%%%%%%%%
\begin{table}[h!]
\begin{tabular}{|
>{\columncolor[HTML]{EFEFEF}}c |l|}
\hline
\cellcolor[HTML]{B9E3F0}\textbf{RF-05} & \multicolumn{1}{c|}{\cellcolor[HTML]{B9E3F0}\textbf{Comunicar una postura incorrecta}}                                                                                                                                                                                \\ \hline
Descripción                            & \begin{tabular}{p{10.5cm}}La solución debe poder comunicar mediante, vibración, sonido u otra manera una mala postura continuada durante un periodo de tiempo definido.\end{tabular} \\ \hline
Importancia                            & \cellcolor[HTML]{FFFFFF}\begin{tabular}{p{10.5cm}}Alta, es necesario que el usuario conozca en todo momento su situación, para poder corregir su postura cuando sea necesario.\end{tabular}                                                                               \\ \hline
Prioridad                              & \multicolumn{1}{c|}{Media}                                                                                                                                                                                                                       \\ \hline
\end{tabular}
\caption{Requisito Funcional 5 'Comunicar una postura incorrecta'}
\label{RF_05}
\end{table}

%%%%%%%%%%%%%%%%%%%%%%%%%%%%% RF-06 %%%%%%%%%%%%%%%%%%%%%%%%%%%%
\begin{table}[h!]
\begin{tabular}{|
>{\columncolor[HTML]{EFEFEF}}c |l|}
\hline
\cellcolor[HTML]{B9E3F0}\textbf{RF-06} & \multicolumn{1}{c|}{\cellcolor[HTML]{B9E3F0}\textbf{Realizar seguimiento}}                                                                                                                                                                                \\ \hline
Descripción                            & \begin{tabular}{p{10.5cm}}La información registrada por el dispositivo debe quedar almacenada para valorar y evaluar la postura del paciente, con el fin de modificar o no el tratamiento o fisioterapia o tomar otro tipo de decisiones.\\ La visualización de la información recogida se reflejará en forma de gráficos y tablas. Esto permitirá analizar la información de manera clara y sencilla
\end{tabular} \\ \hline
Importancia                            & \cellcolor[HTML]{FFFFFF}\begin{tabular}{p{10.5cm}}Alta, ya que será clave para la toma de decisiones por parte del especialista en cuanto a la personalización del tratamiento y rehabilitación.\end{tabular}                                                                               \\ \hline
Prioridad                              & \multicolumn{1}{c|}{Alta}                                                                                                                                                                                                                       \\ \hline
\end{tabular}
\caption{Requisito Funcional 6 'Realizar seguimiento'}
\label{RF_06}
\end{table}

%%%%%%%%%%%%%%%%%%%%%%%%%%%%% RF-07 %%%%%%%%%%%%%%%%%%%%%%%%%%%%
\begin{table}[h!]
\begin{tabular}{|
>{\columncolor[HTML]{EFEFEF}}c |l|}
\hline
\cellcolor[HTML]{B9E3F0}\textbf{RF-07} & \multicolumn{1}{c|}{\cellcolor[HTML]{B9E3F0}\textbf{Manual de usuario}}                                                                                                                                                                                \\ \hline
Descripción                            & \begin{tabular}{p{10.5cm}}La solución deberá incluir unas instrucciones que se entreguen al usuario que lo vaya a utilizar. Esto supone un apoyo durante todo el proceso de uso del dispositivo y de la aplicación por parte del usuario.\end{tabular} \\ \hline
Importancia                            & \cellcolor[HTML]{FFFFFF}\begin{tabular}{p{10.5cm}}Media, puesto que supone un apoyo para el usuario que lo utilice.\end{tabular}                                                                               \\ \hline
Prioridad                              & \multicolumn{1}{c|}{Baja}                                                                                                                                                                                                                       \\ \hline
\end{tabular}
\caption{Requisito Funcional 7 'Manual de usuario'}
\label{RF_07}
\end{table}

%%%%%%%%%%%%%%%%%%%%%%%%%%%%% RF-08 %%%%%%%%%%%%%%%%%%%%%%%%%%%%
\begin{table}[]
\begin{tabular}{|
>{\columncolor[HTML]{EFEFEF}}c |l|}
\hline
\cellcolor[HTML]{B9E3F0}\textbf{RF-08} & \multicolumn{1}{c|}{\cellcolor[HTML]{B9E3F0}\textbf{Batería}}                                                                                                                                                                                \\ \hline
Descripción                            & \begin{tabular}{p{10.5cm}}El dispositivo debe disponer de una batería para poder utilizarlo de forma telemática. Además, la batería del dispositivo debe ser suficiente para el uso previsto.\end{tabular} \\ \hline
Importancia                            & \cellcolor[HTML]{FFFFFF}\begin{tabular}{p{10.5cm}}Media, se debe incluir para mayor comodidad y libertad del paciente al utilizar el dispositivo.\end{tabular}                                                                               \\ \hline
Prioridad                              & \multicolumn{1}{c|}{Media}                                                                                                                                                                                                                       \\ \hline
\end{tabular}
\caption{Requisito Funcional 8 'Batería'}
\label{RF_08}
\end{table}

\clearpage
\subsection{Requisitos no funcionales}
\begin{itemize}
    \item \textbf{Accesibilidad}: la aplicación debe ser accesible para el mayor grupo de personas posible, tengan o no algún tipo de discapacidad.
    \item \textbf{Seguridad}: el dispositivo electrónico debe ser seguro y la información manejada en la aplicación debe estar protegida.
    \item \textbf{Compatibilidad}: la aplicación debe ser compatible con distintos dispositivos.
    \item \textbf{Eficiencia}: la aplicación debe permitir al usuario lograr sus objetivos, con un coste computacional y temporal bajo.
    \item \textbf{Efectividad}: la aplicación debe cumplir con exactitud los requisitos funcionales. 
    \item \textbf{Errores}: la aplicación debe presentar una tasa de error baja, además debe mostrar posibles soluciones en caso de anomalías.
    \item \textbf{Aprendizaje}: tanto el uso del dispositivo electrónico como de la aplicación debe ser sencillo, es decir, se debe poder usar de forma intuitiva.
    \item \textbf{Memorabilidad}: tanto el funcionamiento del dispositivo electrónico como de la aplicación debe ser fácil de recordar, tras no haberlos utilizado durante un tiempo.
    \item \textbf{Satisfacción}: el usuario debe estar satisfecho con el dispositivo electrónico y la aplicación, tanto por su comodidad, estética y usabilidad.


\end{itemize}

%%%%%%%%%%%%%%%%%%%%%%%%%%%%%%%%%%%%%%
\clearpage
% Otra opción para salto de página \newpage
\section{Instalación / Puesta en marcha}[h!]

Introducir XXXXXXXXXXXXXXXXXXXXXXXXXXXXXXXXx
xX

X

X

X


%%%%%%%%%%%%%%%%%%%%%%%%%%%%%%%%%%%%%%%

\section{Manuales y/o Demostraciones prácticas}[h!]

Introducir XXXXXXXXXXXXXXXXXXXXXXXXXXXXXXXXx




    
     