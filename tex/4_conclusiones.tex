\capitulo{4}{Conclusiones}
\textcolor{red}{Todo proyecto debe incluir las conclusiones que se derivan de su desarrollo. Éstas pueden ser de diferente índole, dependiendo de la tipología del proyecto, pero normalmente van a estar presentes un conjunto de conclusiones relacionadas con los resultados del proyecto y un conjunto de conclusiones técnicas. }

\textcolor{red}{Creo que en conclusiones tienes que indicar que al ser la primera vez que se realiza un TFG de estas características también sirve para ver las limitaciones, dificultades y carencias con las uqe un alumno puede enfrentarse al tratar de realizar este tipo de proyectos sin integrarse en un laboratorio en concreto. }

\textcolor{red}{Destacaría la importancia de que es el punto de partida para conseguir colaborar con asociaciones como Parkinson a partir de necesidades reales y concretas mostradas por sus usuarios.}

\section{Resumen de resultados.}

\textcolor{red}{Breve resumen de los resultados. En caso de ser un trabajo muy experimental, los resultados completos pueden aparecer en su anexo correspondiente.}

\section{Discusión.}
\textcolor{red}{Discusión y análisis de los resultados obtenidos.}


\section{Aspectos relevantes.}
\textcolor{red}{
Este apartado pretende recoger los aspectos más interesantes del \textbf{desarrollo del proyecto}, comentados por los autores del mismo.}

\textcolor{red}{Debe incluir los detalles más relevantes en cada fase del desarrollo, justificando los caminos tomados, especialmente aquellos que no sean triviales. }

\textcolor{red}{Puede ser el lugar más adecuado para documentar los aspectos más interesantes del proyecto y también los resultados negativos obtenidos por soluciones previas a la solución entregada.}

\textcolor{red}{Este apartado, debe convertirse en el resumen de la experiencia práctica del proyecto, y por sí mismo justifica que la memoria se convierta en un documento útil, fuente de referencia para los autores, los tutores y futuros alumnos.}

\textcolor{red}{En este punto se podrían incluir los problemas que han surgido durante el desarrollo del proyecto.}

\textcolor{red}{Se han realizado diferentes versiones del prototipo a modo de kit, una primera versión con un sensor tilt (el sensor SW520D). }
\textcolor{red}{La segunda versión del prototipo se ha realizado con el módulo MPU6050.}




