\apendice{Descripción de adquisición y tratamiento de datos}

Como se ha mencionado en la memoria en este proyecto se han empleado \textbf{datos de elaboración propia}, no se han obtenido datos externos de bases de datos existentes. 

Por un lado se trabaja con los datos personales proporcionados por el usuario:
\begin{itemize}
    \item Nombre
    \item Apellidos
    \item Edad
    \item Correo electrónico
    \item Nombre de usuario
    \item Contraseña
\end{itemize}

Por el otro lado, en la segunda versión del prototipo también se trabaja con los datos recopilados por el dispositivo. Se realiza una medición cada 0,5 segundos\footnote{En el futuro se puede modificar y se puede trabajar con medidas medias cada cierto tiempo.} y en cada medición se obtienen los siguientes datos:
\begin{itemize}
    \item Aceleración eje X en $m/s^{2}$.
    \item Aceleración eje Y en $m/s^{2}$.
    \item Aceleración eje Z\footnote{esta aceleración deberá aproximarse a la gravedad de la tierra, a 9.81 $m/s^{2}$} en $m/s^{2}$.
    \item Inclinación eje X en $deg$.
    \item Inclinación eje Y en $deg$.
    \item Rotación eje X en $deg/s$.
    \item Rotación eje Y en $deg/s$.
    \item Rotación eje Z en $deg/s$.
\end{itemize}

Las inclinaciones son las que se emplean en el segundo prototipo para diferenciar entre una postura correcta o incorrecta.

La acumulación de estos datos en bases de datos abrirá el camino a la realización de estadísticas que aporten más información al usuario sobre el estado de su postura y su evolución al emplear el dispositivo.

Todos estos datos deberán ser protegidos conforme a la legislación español definida en el \textit{Anexo A}.

De igual modo, en el futuro, se pueden acumular datos de gran cantidad de personas y de esta forma también se podrían realizar estudios estadísticos con diferentes cohortes para obtener más información sobre la postura y el control postural.
